\section{Remarques}
\subsection{Attention!}
\begin{enumerate}
	\item Lire \textbf{TOUS} les énoncés avant de commencer la moindre implémentation
	\item Faire attention au copier-coller bête et méchant.
	\item Surveiller les overflow. Parfois, un long peut régler pas mal de problèmes
\end{enumerate}
 
\subsection{Opérations sur les bits}
\begin{enumerate}
	\item Vérification parité de $n$: \lstinline{(n & 1) == 0}
	\item $2^n$: \lstinline|1 << n|.
	\item Tester si le $i$ème bit de $n$ est $0$: \lstinline{(n & 1 << i) != 0}
	\item Mettre le $i$ème bit de $n$ à 0: \lstinline{n &= ~(1 << i)}
	\item Mettre le $i$ème bit de $n$ à 1: \lstinline{n |= (1 << i)}
	\item Union: \lstinline{a | b}
	\item Intersection: \lstinline{a & b}
	\item Soustraction bits: \lstinline{a & ~b}
	\item Vérifier si $n$ est une puissance de 2: \lstinline{(n & (n-1) == 0)}
	\item Least significant bit non nul de $n$: \lstinline{(n & (-n))}
	\item Passage au négatif: \lstinline{0x7fffffff^n}
\end{enumerate}

\subsection{Table des complexités}
\begin{center}
\begin{tabular}{|c|c|}
\hline
n $\leq$ & Complexité max\\
\hline
$[10,11]$ & $O(n!),O(n^6)$ \\
$[15,18]$ & $O(2^n n^2)$\\
$[18,22]$ & $O(2^n n)$\\
$100$ & $O(n^4)$\\
$400$ & $O(n^3)$\\
$2K$ & $O(n^2\log(n))$\\
$10K$ & $O(n^2)$\\
$1M$ & $O(n\log(n))$\\
$10M$ & $O(n),O(\log(n)),O(1)$\\
\hline
\end{tabular}
\end{center}